\documentclass[parskip=full]{scrartcl} % enables german formatting

\usepackage[utf8]{inputenc} % use utf8 file encoding for TeX sources
\usepackage[T1]{fontenc} % avoid garbled Unicode text in pdf
\usepackage[german]{babel} % german hyphenation, quotes, etc
\usepackage{hyperref} % detailed hyperlink/pdf configuration
\hypersetup{ % ‘texdoc hyperref‘ for options
pdftitle={PSE},
bookmarks=true,
}
\usepackage{csquotes} % provides \enquote{} macro for "quotes"

\title{Pflichtenheft}
\author{...}
\date{November 2022}

\begin{document}




\maketitle
\newpage

\tableofcontents
\newpage




\section{Zielbestimmung}

\subsection{Musskriterien}
\subsection{Sollkriterien}
\subsection{Kannkriterien}
\subsection{Abgrenzungskriterien}
\newpage



\section{Produkteinsatz}

\subsection{Anwendungsbereiche}
\subsection{Zielgruppen}
In diesem Abschnitt wird die Zielgruppe des Produkts definiert. Prinzipiell kann jede Person die Anwendung nutzen. Nutzer müssen jedoch Grundkenntnisse im OSM-Dateiformat besitzen. Außerdem sind die Anwendungsbereiche, wie oben beschrieben, beschränkt. Daher bilden Verkehrsingenieure und Verkehrswissenschaftler die größte Zielgruppe des Produkts. Da die Zielgruppe insbesondere nicht aus Informatikern besteht, werden keine Kenntnisse in der Programmierung vorausgesetzt.

\subsection{Betriebsbedingungen}
\newpage



\section{Produktübersicht}
\newpage



\section{Produktfunktionen}
\newpage



\section{Produktdaten}
\newpage



\section{Nichtfunktionale Anforderungen}
Hier möglicherweise noch Unterkapitel
\begin{enumerate}
    \item Funktionalität
    \item Sicherheit
    \item Benutzbarkeit
    \item Änderbarkeit
    \item Qualitätsanforderungen
\end{enumerate}
\newpage



\section{Benutzeroberfläche/Schnittstellen}
\newpage



\section{Technische Produktumgebung}

\subsection{Software}
\subsection{Hardware}
\subsection{Produktschnittstellen}
\newpage




\end{document}