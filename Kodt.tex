\documentclass[parskip=full]{scrartcl} % enables german formatting

\usepackage[utf8]{inputenc} % use utf8 file encoding for TeX sources
\usepackage[T1]{fontenc} % avoid garbled Unicode text in pdf
\usepackage[german]{babel} % german hyphenation, quotes, etc
\usepackage{hyperref} % detailed hyperlink/pdf configuration
\usepackage{setspace} % to set spacings between lines
\usepackage{parskip}
\hypersetup{ % ‘texdoc hyperref‘ for options
pdftitle={PSE},
bookmarks=true,
}
\usepackage{csquotes} % provides \enquote{} macro for "quotes"

\title{Pflichtenheft}
\author{Karl, JP, Simon, Felix, Pascal}
\date{November 2022}

\begin{document}




\maketitle
\newpage

\tableofcontents
\newpage




\section{Zielbestimmung}

\textbf{author=Pascal,not reviewed yet\\}
Verkehrsmodelle sind allgegenwärtig, egal ob man nur ein Teilnehmer am Verkehr ist oder Entscheidungen treffen muss um diesen zu optimieren.
Für eine effektive Optimierung des Verkehrswesens sind Daten über den Verkehr unumgänglich und abgesehen von der geografischen Lage von Gebäuden, Straßen usw. ist auch das Verhalten der Menschen ein wichtiger Bestandteil.
Um diese Menschen effektiv simulieren zu können, bedarf es Spezifikation und Konfiguration geografischer Daten um ein verhalten von Menschen berechenbar zu machen.
Hier kommt die (NAME DER SOFTWARE) ins spiel.
Mit Hilfe von (NAME DER SOFTWARE) soll es  Verkehrsingenieuren und Verkehrswissenschafter möglich sein, genau solche Daten für ihre Berechnungen anfertigen zu können.
Hierbei werden Daten aus der OSM als Basis genutzt.
Die Nutzer müssen nun kein Vorwissen an Informatik mit sich bringen um die Software bedienen zu können.
Die Software simplifiziert die OSM Daten von Gebäude auf Punkte. Der Nutzer kann diese mit Tags versehen und Kategorisieren über selber angegebene Filter.
Die Berechnung für Attraktivität soll Konfigurierbar sein und kann von beliebigen Attributen des Nutzers festgelegt werden.
Neben der Attraktivität einzelner Elemente kann auch über mehrere Aggregiert werden, was ebenfalls Konfigurierbar sein soll.

\subsection{Musskriterien}
Musskriterien(kurz MK): unabdingbare Leistungen der Software.

\begin{itemize}
    \item <MK1> Der Nutzer muss Projekte erstellen können
    \item <MK2> Der Nutzer muss OSM-Datensätze einlesen können.
    \item <MK3> Der Nutzer muss Kartenausschnitte auswählen können.
    \item <MK4> Der Nutzer muss OSM-Elemente mithilfe von Black- und Whitelisten zu Kategorien zusammenfassen können.
    \item <MK5> Das Programm muss Standard-Kategorien anbieten.
    \item <MK6>Das Programm muss OSM-Elemente auf einen Punkt reduzieren können.
    \item <MK7> Das Programm muss den OSM-Elementen Attribute zuweisen können.
    \item <MK8> Der Nutzer muss für jede Kategorie das Verfahren zur Berechnung der Attribute auswählen können.
    \item <MK9> Der Nutzer muss Standardwerte für Attribute festlegen können.
    \item <MK10> Der Nutzer muss Attraktivitätsattribute je Kategorie in Abhängigkeit zu Attributen definieren können.
    \item <MK11> Das Programm muss in jeder Verkehrszelle die Attraktivitätsattribute aggregieren können.
    \item <MK12> Das Programm muss die Zwischenschritte der Berechnungen und Konfigurationen speichern und laden können.
    \item <MK13> Der Nutzer muss die Berechnungen aus jeder Phase aus starten können.
    \item <MK14> Es muss eine grafische Oberfläche geben.
\end{itemize}


\subsection{Wunschkriterien}
\textbf{author=Simon, reviewedBy=Karl, reviewDone = True\\}
Wunschkriterien(kurz WK): wünschenswerte Leistungen der Software.
\begin{itemize}
    \item <WK1> Der Nutzer kann OSM-Daten per Link oder OSM-Repository einlesen.
    \item <WK2> Die Verkehrsdaten können mit einer Karte visualisiert werden.
    \item <WK3> Der Nutzer kann durch Klicken auf eine Karte Polygone definieren.
    \item <WK4> Es kann für das Filtern mit Kategorien beliebige logische Ausdrücke möglich sein.
    \item <WK5> Das Programm bietet eine Liste an OSM-Tags an
    \item <WK6> Das Programm kann die Ergebnisse der Aggregation grafisch darstellen
    \item <WK7> Die Anwendung markiert Phasen, falls sie vom Nutzer abgeändert werden
    \item <WK8> Die Anwendung besitzt eine Statusanzeige, die den aktuellen Fortschritt der Berechnungen anzeigt
    \item <WK9> Der Nutzer kann die Attraktivitätsattribute auch in Abhängigkeit zu OSM-Tags definieren.
    
\end{itemize}

\subsection{Abgrenzungskriterien}
\begin{itemize}
    \item Es findet keine Analyse der generierten Daten statt.
    \item Die Anwendung ist nicht für Mobilgeräte optimiert.
    \item Es findet keine Analyse der Straßennetze statt.
\end{itemize}
\newpage

\section{Produkteinsatz}

\subsection{Anwendungsbereiche}
\textbf{author="JP", reviewedby=Karl} 
Dieses Kapitel erläutert, in welchen Bereichen das Produkt eingesetzt werden soll. Der Anwendungsbereich des Produkts liegt bei Verkerhrsinstituten. Diese Nutzen die Berechnung von Attratktivitäten verschiedener Verkehrszellen zum Erstellen von Verkehrsnachfragemodellen. Diese sind später die Grundlage, um sowohl verkehrsplanerische, wie auch politische Entscheidungen bezüglich der Infrastruktur zu treffen.


\subsection{Zielgruppen}
\textbf{reviewedby = Felix, JP author=Karl\\}
Dieser Abschnitt definiert die Zielgruppe des Projekts. Prinzipiell kann jede Person die Anwendung nutzen. Nutzer müssen jedoch Grundkenntnisse im OSM-Dateiformat besitzen. Die Anwendungsbereiche sind, wie oben beschrieben, beschränkt. Daher bilden Verkehrsingenieure und Verkehrswissenschaftler die größte Zielgruppe des Produkts. Da die Zielgruppe insbesondere nicht aus Informatikern besteht, werden keine Kenntnisse in der Programmierung vorausgesetzt.


\newpage







\section{Produktübersicht}
\newpage










\section{Produktfunktionen}
Blabla, Erklärtext
\newpage

\subsection*{Vorlage für Steckbrief}
\textbf{Anwendungsfall:} ...\\\\
\textbf{Anforderungen:} ...\\\\
\textbf{Ziel:} A \\\\
\textbf{Vorbedingung:} B \\\\
\textbf{Nachbedingung Erfolg:} C \\\\
\textbf{Nachbedingung Fehlschlag:} D \\\\
\textbf{Auslösendes Ereignis:} E \\\\
\textbf{Beschreibung:}
\begin{enumerate}
    \item F
    \item G
    \item H
\end{enumerate}
\textbf{Erweiterungen:} 
\begin{itemize}
    \item 3a: I
\end{itemize}
\textbf{Alternativen:} 
\begin{itemize}
    \item 3a: J
\end{itemize}
\newpage


\subsection*{Starten des Programms <A1>}
\textbf{Anwendungsfall:}  Der user möchte das Programm starten\\\\
\textbf{Anforderungen:} ...\\\\
\textbf{Ziel:} Das Programm wurde gestartet und läuft \\\\
\textbf{Vorbedingung:} Das Programm muss installiert sein  \\\\
\textbf{Nachbedingung Erfolg:} Das Programm läuft erfolgreich, die `Main-Menu` Seite wird angezeigt\\\\
\textbf{Nachbedingung Fehlschlag:} Das Programm startet nicht \\\\
\textbf{Auslösendes Ereignis:} Der User führt das Programm aus \\\\
\textbf{Beschreibung:}
\begin{enumerate}
    \item Der User führt das Programm aus
    \item Die `Main-Menu` Seite wird angezeigt
\end{enumerate}
\newpage


\subsection*{Neues Projekt erstellen <A2>}
\textbf{Anwendungsfall:} Der user möchte ein neues Projekt erstellen\\\\
\textbf{Anforderungen:} ...\\\\
\textbf{Ziel:} Ein neues Projekt wurde erstellt \\\\
\textbf{Vorbedingung:} Das Programm muss auf der `Main-Menu` Seite sein   \\\\
\textbf{Nachbedingung Erfolg:} Ein neues Projekt wurde im Speicher erstellt  \\\\
\textbf{Nachbedingung Fehlschlag:} Benachrichtigung des Nutzers über Fehlschlag, Programm  bleibt in der `Main-Menu` Seite \\\\
\textbf{Auslösendes Ereignis:} Der User hat auf den `Create New Project` Knopf gedrückt \\\\
\textbf{Beschreibung:}
\begin{enumerate}
    \item Der User wählt den Namen und die Beschreibung für das Projekt
    \item Der User drückt auf den `Create` Knopf
    \item Es wird ein Projekt Ordner im Speicherort erstellt, den der User selbst wählen kann oder nicht
    \item Es wird eine Projekt Konfiguration Datei erstellt und im Projek-tOdner abgespeichert
\end{enumerate}
\newpage


\subsection*{Internes Projekt laden <A3>}
\textbf{Anwendungsfall:} Der User möchte ein Projekt aus dem Standart-Ordner laden\\\\
\textbf{Anforderungen:} ...\\\\
\textbf{Ziel:} Ein Projekt das schon in dem dem Standart-Ordner existiert wurde geladen \\\\
\textbf{Vorbedingung:} Das Programm muss auf der `Main-Menu` Seite sein  \\\\
\textbf{Nachbedingung Erfolg:} Alle Daten des Projekts wurden geladen und die `Daten` Seite wird angezeigt \\\\
\textbf{Nachbedingung Fehlschlag:} Benachrichtigung des Nutzers über Fehlschlag, Programm  bleibt in der `Main-Menu` Seite \\\\
\textbf{Auslösendes Ereignis:}  Der User hat ein Projekt ausgewählt aus dem Standart-Speicher-Ordner und den `Load internal Project` Knopf gedrückt \\\\
\textbf{Beschreibung:}
\begin{enumerate}
    \item Das Projekt wird intern geladen
\end{enumerate}
\newpage

\subsection*{Externes Projekt laden <A4>}
\textbf{Anwendungsfall:} Der User möchte ein Projekt aus einem nicht Standart-Ordner laden\\\\
\textbf{Anforderungen:} ...\\\\
\textbf{Ziel:} Ein Projekt das nicht in dem dem Standart-Ordner existiert wurde geladen \\\\
\textbf{Vorbedingung:} Das Programm muss auf der `Main-Menu` Seite sein  \\\\
\textbf{Nachbedingung Erfolg:} Alle Daten des Projekts wurden geladen und die `Daten` Seite wird angezeigt \\\\
\textbf{Nachbedingung Fehlschlag:} Benachrichtigung des Nutzers über Fehlschlag, Programm  bleibt in der `Main-Menu` Seite \\\\
\textbf{Auslösendes Ereignis:}  Der User hat den `Load external Project` Knopf gedrückt \\\\
\textbf{Beschreibung:}
\begin{enumerate}
    \item Der Datei-Explorer öffnet sich.
    \item Der Nutzer wählt eine Datei im passenden Format aus.
\end{enumerate}
\newpage


\subsection*{Dateien referenzieren <A5>}
\textbf{Anwendungsfall:} ...\\\\
\textbf{Anforderungen:} ...\\\\
\textbf{Ziel:} Der Nutzer definiert die OSM-Daten und den räumlichen Filter\\\\
\textbf{Vorbedingung:} Der Nutzer befindet sich auf der Seite 'Data'. \\\\
\textbf{Nachbedingung Erfolg:} Erfolgreiches Hinterlegen der angegebenen Dateien. Anzeigen der angegebenen Dateien.\\\\
\textbf{Nachbedingung Fehlschlag:} Benachrichtigung des Nutzers über Fehlschlag. \\\\
\textbf{Auslösendes Ereignis:} Drücken des Knopfes 'Select OSM-Data'\\\\
\textbf{Beschreibung:}
\begin{enumerate}
    \item Der Datei-Explorer öffnet sich.
    \item Der Nutzer wählt eine Datei im passenden Format aus.
    \item Die Anwendung zeigt den Namen der gewählten Datei an.
    \item Der Nutzer drückt auf den Knopf 'Select Cut-Out'
    \item Der Datei-Explorer öffnet sich.
    \item Der Nutzer wählt eine Datei im passenden Format aus.
    \item Die Anwendung zeigt den Namen der gewählten Datei an.
    \item Der Nutzer wählt eine der folgenden Optionen aus
    \begin{itemize}
        \item Im räumlichen Filter befinden sich nur OSM-Elemente, welche sich vollständig in den angegebenen Verkehrszellen befinden.
        \item Im räumlichen Filter befinden sich auch OSM-Elemente, welche sich nur teilweise in den angegebenen Verkehrszellen befinden
    \end{itemize}
\end{enumerate}
\newpage


% Noch nicht fertig, wir müssen noch besprechen, wo das stattfindet
\subsection*{Kategorien importieren <A7>}
\textbf{Anwendungsfall:} Der Nutzer möchte in seinem Projekt Kategorien verwenden, welche bereits in einem anderen definiert wurden.\\\\
\textbf{Anforderungen:} ...\\\\
\textbf{Ziel:} Das Importieren vordefinierter Kategorien. \\\\
\textbf{Vorbedingung:} Der Nutzer befindet sich auf der Seite 'Daten' \\\\
\textbf{Nachbedingung Erfolg:} Die Kategorien der importierten Datei sind in der Liste der Kategorien aufgelistet \\\\
\textbf{Nachbedingung Fehlschlag:} Die Anwendung informiert den Nutzer über den Fehlschlag. \\\\
\textbf{Auslösendes Ereignis:} Drücken des Knopfes 'Import categories'\\\\
\textbf{Beschreibung:}
\begin{enumerate}
    \item Der Datei-Explorer öffnet sich.
    \item Der Nutzer wählt eine Datei im passenden Format aus
    \item Die Anwendung zeigt den Namen der ausgewählten Datei an.
    \item Der Nutzer bestätigt den Import der Kategorien.
    \item Die Anwendung fügt die Kategorien zur Kategorienliste hinzu
\end{enumerate}
\textbf{Erweiterungen:} 
\begin{itemize}
    \item 3a: Neben dem Namen der Datei werden auch die in ihr enthaltenen Daten visualisiert.
\end{itemize}
\textbf{Alternativen:} 
\begin{itemize}
    \item 5a: Die Anwendung ersetzt die bisherigen Kategorien mit den importierten.
\end{itemize}
\newpage



\subsection*{Erstellen von Kategorien <A9>}
\textbf{Anwendungsfall:} Der Nutzer möchte eine weitere Kategorie hinzufügen\\\\
\textbf{Anforderungen:} ...\\\\
\textbf{Ziel:} Die Liste der Kategorien soll um eine weitere Kategorie ergänzt werden \\\\
\textbf{Vorbedingung:} Der Nutzer befindet sich auf der Seite 'Kategorien' \\\\
\textbf{Nachbedingung Erfolg:} In der Liste der Kategorien befindet sich eine neue, nicht-konfigurierte Kategorie. Die neue Kategorie ist ausgewählt.\\\\
\textbf{Auslösendes Ereignis:} Der Nutzer drückt auf den Button zum Hinzufügen einer neuen Datei. \\\\
\textbf{Beschreibung:}
\begin{enumerate}
    \item Die Anwendung schreibt einen leeren Eintrag in die Kategorienliste
    \item Die Anwendung wählt den neuen Eintrag aus
\end{enumerate}
\newpage



\subsection*{Konfgurieren von Kategorien <A10>}
\textbf{Anwendungsfall:} Der Nutzer möchte die Eigenschaften einer Kategorie verändern\\\\
\textbf{Anforderungen:} ...\\\\
\textbf{Ziel:} Konfigurieren einer Kategorie \\\\
\textbf{Vorbedingung:} Der Nutzer befindet sich auf der Seite 'Kategorien'. Es existiert mindestens eine Kategorie.\\\\
\textbf{Nachbedingung Erfolg:} Die Änderungen an der Kategorie und an ihren Eigenschaften wird angezeigt. \\\\
\textbf{Nachbedingung Fehlschlag:} Die Anwendung zeigt den Grund für den Fehlschlag an. \\\\
\textbf{Auslösendes Ereignis:} Der Nutzer klickt auf eine Kategorie oder der Nutzer erstellt eine neue Kategorie \\\\
\textbf{Beschreibung:}
\begin{enumerate}
    \item Die Anwendung zeigt dem Nutzer den Namen, die Whitelist und die Blacklist der Kategorie an. Die Anwendung signalisiert, ob die Kategorie aktiviert ist.
    \item Der Nutzer kann keine, eine oder mehrere der folgenden Aktionen durchführen:
    \begin{itemize}
        \item Der Nutzer kann die Kategorie aktivieren oder deaktivieren.
        \item Der Nutzer kann den Namen der Anwendung ändern.
        \item Der Nutzer kann Einträge zur Blacklist und zur Whitelist hinzufügen
    \end{itemize}
\end{enumerate}
\newpage





\subsection*{Phase wechseln}
\textbf{Anwendungsfall:} Der Nutzer möchte in eine andere Phase der Konfigurierung wechseln.\\\\
\textbf{Anforderungen:} ...\\\\
\textbf{Ziel:} Die aktuelle Konfigurationsseite wird gespeichert und eine andere geladen. \\\\
\textbf{Vorbedingung:} Ein Projekt ist geladen. \\\\
\textbf{Nachbedingung Erfolg:} Die ausgewählte Seite wird angezeigt. Änderungen der alten Seite sind gespeichert.\\\\
\textbf{Nachbedingung Fehlschlag:} Die alte Seite wird weiterhin angezeigt. Die Anwendung informiert den Nutzer über den Fehlschlag. \\\\
\textbf{Auslösendes Ereignis:} Drücken auf eine Phase im Kopf der Seite oder Drücken auf einen 'Continue' Button \\\\
\textbf{Beschreibung:}
\begin{enumerate}
    \item Die Anwendung speichert die Daten, die der Nutzer auf der aktuellen Seite hinterlegt hat.
    \item Die Anwendung lädt die Daten für die ausgewählte bzw. nächste Seite
    \item Die Anwendung öffnet die ausgewählte bzw. nächste Seite
\end{enumerate}
\textbf{Erweiterungen:} 
\begin{itemize}
    \item 3a: Die Anwendung informiert den Nutzer über die Speicherung der Daten
\end{itemize}
\newpage



\subsection*{Anwendung beenden}
\textbf{Anwendungsfall:} Der Nutzer möchte die Anwendung schließen\\\\
\textbf{Anforderungen:} ...\\\\
\textbf{Ziel:} Die Anwendung soll sich schließen und nicht-gespeicherte Daten speichern \\\\
\textbf{Vorbedingung:} Die Anwendung ist gestartet\\\\
\textbf{Nachbedingung Erfolg:} Die Anwendung ist geschlossen und nicht-gespeicherte Daten sind gespeichert\\\\
\textbf{Nachbedingung Fehlschlag:} Die Anwendung ist noch geöffnet \\\\
\textbf{Auslösendes Ereignis:} Drücken auf das Kreuz, zum Beenden der Anwendung, welches durch das Betriebssystem üblich ist. \\\\
\textbf{Beschreibung:}
\begin{enumerate}
    \item Befindet sich der Nutzer auf einer Konfigurationsseite mit nicht-gespeicherten Daten, so werden diese zwischengespeichert.
    \item Die Anwendung schließt sich
\end{enumerate}
\textbf{Erweiterungen:} 
\begin{itemize}
    \item 1a: Gibt es nicht-gespeicherte Daten, so wird der Nutzer aufgefordert diese zu speichern oder zu verwerfen.
\end{itemize}
\newpage









\section{Produktdaten}
Dieses Kapitel beschreibt die Organisation der Produktdaten. Daten, welche nicht unbedingt erhoben werden müssen, sind mit einem Stern ($\ast$) markiert. Die größte Organisationseinheit sind hierbei die Projekte. Jedem Projekt ist ein Verzeichnis zugeordnet. Dieses Verzeichnis wird im folgenden auch als Projektordner bezeichnet. In jedem Projektordner befinden sich die folgenden Dateien und Verzeichnisse:
\begin{itemize}
    \item Projekt-Konfiguration (JSON-Datei)
    \item Kategorien-Konfiguration (JSON-Datei)
    \item Zwischenergebnisse (Verzeichnis)
    \item Ausgabe (Verzeichnis)
\end{itemize}

\subsection*{Projekt-Konfiguration}
Diese JSON-Datei enthält die wichtigsten Grundeinstellungen des jeweiligen Projektes. Darunter:
\begin{itemize}
    \item Name des Projekts
    \item Beschreibung des Projekts ($\ast$)
    \item Erstellungsdatum des Produkts ($\ast$)
    \item Datum der letzten Änderung ($\ast$)
    \item Definition des räumlichen Filters: Müssen OSM-Elemente vollständig im definierten Bereich liegen, um betrachtet zu werden?
    \item Liste der Statistiken, die während der Aggregation berechnet werden sollen
    \item Verweis auf die OSM-Eingabedatei
    \item Verweis auf die Datei, welche die Verkehrszellen definiert
\end{itemize}


\subsection*{Kategorien-Konfiguration}
Diese JSON-Datei enthält alle Informationen zu den Kategorien des jeweiligen Projekts. Es folgt eine Liste der Eigenschaften, die je Kategorie gespeichert werden müssen.
\begin{itemize}
    \item Name der Datei (eindeutig)
    \item Einträge der Tag-Whitelist
    \item Einträge der Tag-Blacklist
    \item Definition der Default-Werte je Attribut
    \item Konfiguration der Punktreduktion
    \item Faktoren der Attribute je Attraktivitätsattribut
\end{itemize}


\subsection*{Zwischenergebnisse}
Dieses Verzeichnis enthält die Zwischenergebnisse der Berechnungen der jeweiligen Phasen.

\subsection*{Ausgabe}
Dieses Verzeichnis enthält die finale Ausgabe der Berechnungen. Hier befindet sich je ausgewählter Aggregationsstatistik eine CSV-Datei. Jede dieser CSV-Dateien ordnet jeder Verkehrszelle je Attraktivitätsattribut einen numerischen Wert zu.


\subsection*{Projektverwaltung}
Der Nutzer kann Projekte bzw. ihre Projektordner grundsätzlich an beliebigen Stellen speichern. Jedoch gibt es einen Projektstandardordner. In diesem werden Projekte standardmäßig gespeichert. Die GUI ermöglicht ein nutzerfreundlicheres Öffnen von Projekten aus dem Projektstandardordner. Der Nutzer kann den Projektstandardordner beliebig verschieben.

\subsection*{Dateiformate}
Zum Lesen der anwendungsspezifischen Daten verwendet die Anwendung Dateiformate:
\begin{itemize}
    \item OSM-Dateien: Zum Lesen und Speichern von OSM-Elementen unterstützt die Anwendung die Dateiformate .osm.bz2 und .osm.pbz.
    \item Verkehrszellen: Zum definieren von Verkehrszellen verwendet die Anwendung die Dateiformate shape und geojson.
\end{itemize}


\newpage







\section{Nichtfunktionale Anforderungen}

Dieses Kapitel definiert die nichtfunktionalen Anforderungen und Qualitätsmerkmale der Software.

Reminder: Punktreduktion soll erweiterbar sein



\subsection{Funktionalität, author=Pascal, reviewedby=Karl}

\begin{tabular}{|l| c| c| c| c|}
    \hline
        Produktqualität & sehr gut & gut & normal & nicht relevant \\
    \hline
        Angemessenheit & & x & &\\
    \hline
        Richtigkeit & x & & &\\
    \hline
        Ordnungsmäßgkeit & x & & &\\
    \hline
        Interoperabilität & & & & x\\
    \hline
        
    \end{tabular}

\textbf{Angemessenheit}
\newline
Die Angemessenheit ist als gut einzustufen, da die gesamte Software für ihre Funktion geeignet sein muss und eine schwere Handhabung der Software nur zu unnötigem Aufwand seitens des Nutzers führt.



\textbf{Richtigkeit und Ordnungsmäßigkeit}
\newline
Die Aufgabe der Software ist über die Konfigurationen des Nutzers und den eingelesenen Dateien, neue Daten zu berechnen, welche weiter verwendet werden können. Hierbei ist es wichtig, dass diese Daten auch korrekt berechnet werden und ein deterministisches Verhalten gewährleistet ist.
Daher ist die Richtigkeit und Ordnungsmäßigkeit der Software als sehr gut einzustufen.

\textbf{Interoperabilität}
\newline
Es ist keine Kommunikation zwischen Geräten vorgesehen, daher ist die Interoperabilität als irrelevant einzustufen.


% remove this if we dont want to have a new page for each
% subsection
\newpage 

\subsection{Sicherheit, author=Simon, reviewed by Pascal}

    \begin{tabular}{|l| c| c| c| c|}
    \hline
        Produktqualität & sehr gut & gut & normal & nicht relevant \\
    \hline
        Zuverlässigkeit & x & & &\\
    \hline
        Fehlertoleranz & & x & &\\
    \hline
        Wiederherstellbarkeit & & x & &\\
    \hline
     \end{tabular}

\textbf{Zuverlässigkeit\\}
Die Zuverlässigkeit ist wichtig, da Benutzer erwarten, dass die Anwendung konsistente Ergebnisse liefert.
Es wird erwartet, dass das Programm die eingegebenen Daten basierend auf Benutzereingaben korrekt auswertet.
Durch das Definieren von Testfällen in dem Pflichtenheft und das Ausführen dieser Testfälle während der Implementierungsphase können Fehler frühzeitig erkannt und behoben werden.
Dies gewährleistet eine sehr gute Zuverlässigkeit.

\textbf{Fehlertoleranz\\}
Trotz aller Testfälle, kann es zu Fehlern kommen während das Programm läuft. Diese Fehler können Entstehen durch korrumpierte Daten oder durch fehlerhafte Eingabe des Users.
Fehler die durch falsche Eingabe des Users Entstehen können abgefangen werden und der User benachrichtigt werden, somit kann dieser die fehlerhafte Eingabe korrigieren.
Fehler die ausgelöst werden durch korrumpierte Daten, können behoben werden indem der User benachrichtigt wird und dieser die korrumpierte Daten ersetzt.
Da diese Fehler nur entstehen können durch das falsche Verhalten des Users, kann 
die Fehlertoleranz als gut eingestuft werden.

\textbf{Wiederherstellbarkeit\\}
Beim Erstellen der Konfigurationen kann der Nutzer zu jedem Zeitpunkt die Konfigurationen lokal zwischenspeichern. Diese kann er zu einem späteren Zeitpunkt wieder einlesen. Dadurch wird gewährleistet, dass die Konfigurationsentscheidungen des Nutzers einfach wiederhergestellt werden können.
Auch alle Zwischenergebnisse, der vom Programm durchgeführten Berechnungen, werden direkt nach Fertigstellung auf dem Endgerät gespeichert. Während der einzelnen Berechnungen sind jedoch keine Zwischenspeicherungen vorgesehen.
Somit ist die die Wiederherstellbarkeit als gut einzustufen.


% remove this if we dont want to have a new page for each
% subsection
\newpage 


\subsection{Benutzbarkeit, author = Felix}

    \begin{tabular}{|l| c| c| c| c|}
    \hline
        Produktqualität & sehr gut & gut & normal & nicht relevant \\
    \hline
        Verständlichkeit & x & & &\\
    \hline
        Erlernbarkeit & & x & &\\
    \hline
        Bedienbarkeit & x & & &\\
    \hline
        Effizienz & & & x &\\
    \hline
        Zeitverhalten & & & x &\\
    \hline
        Verbrauchsverhalten & & x & &\\
    \hline
    \end{tabular}

\textbf{Verständlichkeit, Erlernbarkeit und Bedienbarkeit}\\
Die Anwendung soll intuitiv zugänglich sein, um ein reibungsloses Einlesen der Daten zu ermöglichen. Da die Anwendung auch an Nicht-Informatiker gerichtet ist, muss die Bedienung leicht einsehbar und verständlich sein. Dies erfolgt über eine moderne GUI.

\textbf{Effizienz, Zeitverhalten und Verbrauchsverhalten}\\
Die Effizienz der Anwendung ist als normal eingestuft, da kein schnelles Reaktionsverhalten in der Berechnung gefordert ist, trotzdem aber eine gewisse Benutzerfreundlichkeit erhalten bleiben werden muss. Folglich soll die Anwendung ein gutes Zeit- und Verbrauchsverhalten haben.
Eine kleine Instanz (bspw. Stadtkreis Karlsruhe) ist auf einem Laptop mit 8GB RAM innerhalb weniger Stunden berechenbar. Für größere Instanzen wird eine Workstation empfohlen.


% remove this if we dont want to have a new page for each
% subsection
\newpage 


\subsection{Änderbarkeit, author = JP, reviewedBy = Felix}
    \begin{tabular}{|l| c| c| c| c|}
    \hline
        Produktqualität & sehr gut & gut & normal & nicht relevant \\
    \hline
        Analysierbarkeit x & & & &\\
    \hline
        Modifizierbarkeit & x & & &\\
    \hline
        Stabilität & x & & &\\
    \hline
        Prüfbarkeit & & & x &\\
    \hline
        Übertragbarkeit & & & x &\\
    \hline
        Anpassbarkeit & x & & &\\
    \hline
        Installierbarkeit & & & x &\\
    \hline
    \end{tabular}
    
\textbf{Analysierbarkeit, Modifizierbarkeit und Anpassbarkeit}\\
Erweiterungen stellen ein häufig gewünschtes Kriterium dar. Deshalb ist das Produkt gut modifizierbar. Um Änderungen und Erweiterungen, wie das Hinzufügen von Verfahren bei der Reduktion oder Methoden bei der Aggregation, umzusetzen ist jedoch Code-Verständnis erforderlich und diese müssen von einem Informatiker implementiert werden. Die Anpassbarkeit des Produktes ist gut. Änderungen an der Benutzeroberfläche beziehungsweise der Implementierung der Berechnungs-Verfahren beeinflussen sich nicht untereinander. Um dies zu gewährleisten, muss der muss der Programmcode übersichtlich, strukturiert, kommentiert und somit sehr gut analysierbar sein.

\textbf{Stabilität}\\
Die Benutzeroberfläche darf die Funktionalität des Produktes nicht beeinträchtigen. Daher muss eine gute Stabilität gewährleistet werden.

\textbf{Installierbarkeit}\\
Das Produkt ist als Open-Source-Anwendung online erhältlich. Die Installierbarkeit ist als normal einzustufen.


% remove this if we dont want to have a new page for each
% subsection
\newpage 


\subsection{Qualitätsanforderungen, author = Karl, reviewedBy = Felix}
In diesem Abschnitt werden die oben genannten Qualitätseigenschaften als konkrete Qualitätsanforderungen formuliert.

\begin{itemize}
    \item <Q1> Die Anwendung soll in englischer Sprache verfügbar sein
    \item <Q2> Die Anwendung soll intuitiv bedienbar sein
    \item <Q3> Die Anwendung muss auf Windows 10 und 11 nutzbar sein
    \item <Q4> Durch eine große Testabdeckung soll die Korrektheit des Produkts gewährleistet sein
    \item <Q5> Das Produkt soll um neue Verfahren erweiterbar sein
    \item <Q6> Das Produkt muss als Open-Source-Projekt weiterentwickelbar sein
    \item <Q7> Der Quelltext soll lesbar und kommentiert sein
    \item <Q8> Die Berechnungszeiten sollen minimal sein
    \item <Q9> Die Programmiersprache muss Python sein.
    \item <Q10> Der Programmablauf soll deterministisch sein
    \item <Q11> Die Anwendung soll von unerfahrenen Nutzern leicht installierbar sein
\end{itemize}

\newpage



\section{Benutzeroberfläche/Schnittstellen}
\newpage



\section{Technische Produktumgebung, author = JP}
In diesem Kapitel wird die technische Umgebung des Produktes beschrieben.

\subsection{Software}
\begin{itemize}
    \item Implementierungssprache der Anwendung – Python
    \item Implementierungssprache der Benutzeroberfläche - Python (?)
\end{itemize}

\subsection{Hardware}
\begin{itemize}
    \item Windows 10/11 Laptop mit 8 GB RAM und i5 4 Kerne (8 log.) für kleine Instanzen
    \item Workstation mit 125 – 250 GB RAM und 8-16 Kerne für große Instanzen Effiziente
\end{itemize}

\subsection{Produktschnittstellen}
Die Benutzerschnittstelle wird über ein GUI zur Verfügung gestellt.

\newpage
\section{Glossar}

\textbf{Aggregation}\\
Verfahren zum Zusammenfassen mehrerer Attraktivitäten innerhalb einer Verkehrszelle.

\textbf{Attribut}\\

\textbf{Attraktivität}\\
Eigenschaft eines Gebäudes zur Beschreibung, wie gut ein Objekt in OSM basierend auf verschiedenen Faktoren ist. Diese prognostiziert die Beliebtheit des Gebäudes.

\textbf{Attraktivitätsattribut}\\


\textbf{Geofilter}\\
Werkzeug zur Einteilung von Zellen in der Karte.

\textbf{Kategorie}\\


\textbf{Node}\\
Objekt in OSM.

\textbf{OSM}\\
OpenStreetMap ist ein freies Projekt, das frei nutzbare Geodaten sammelt, strukturiert und für die Nutzung durch jedermann in einer Datenbank vorhält.

\textbf{Reduktion}\\
Verfahren zur Abbildung von Gebäuden auf Punkte mit Attraktivitäten.

\textbf{Repository}\\
Zentrale Ablage in der digitale Daten, Dokumente, Objekte und Programme mit ihren Metadaten verwaltet werden.

\textbf{Tag}\\
Eigenschaft einer Node in OSM.

\textbf{Tag Filter}\\
Filterung nach Tags.

\textbf{Verkehrsnachfragemodell}\\
Ein Verkehrsnachfragemodell ist ein Modell, das alle relevanten Entscheidungsprozesse der Menschen nachbildet, die zu Ortsveränderungen führen. Im Personenverkehr umfassen diese Entscheidungen die Aktivitätenwahl, die Zielwahl, die Verkehrsmittelwahl, die Abfahrtszeitwahl und die Routenwahl.

\textbf{Verkehrszelle}\\
Bezeichnet einen räumlich abgegrenzten Teil eines Untersuchungsgebietes, der als Bezugseinheit bei der Verkehrsanalyse und Verkehrsprognose dient.

\end{document}