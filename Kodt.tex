\documentclass[parskip=full]{scrartcl} % enables german formatting

\usepackage[utf8]{inputenc} % use utf8 file encoding for TeX sources
\usepackage[T1]{fontenc} % avoid garbled Unicode text in pdf
\usepackage[german]{babel} % german hyphenation, quotes, etc
\usepackage{hyperref} % detailed hyperlink/pdf configuration
\hypersetup{ % ‘texdoc hyperref‘ for options
pdftitle={PSE},
bookmarks=true,
}
\usepackage{csquotes} % provides \enquote{} macro for "quotes"

\title{Pflichtenheft}
\author{...}
\date{November 2022}

\begin{document}




\maketitle
\newpage

\tableofcontents
\newpage




\section{Zielbestimmung}

\textbf{author=Pascal,not reviewed yet\\}
Verkehrsmodelle sind allgegenwärtig, egal ob man nur ein Teilnehmer am Verkehr ist oder Entscheidungen treffen muss um diesen zu optimieren.
Für eine effektive Optimierung des Verkehrswesens sind Daten über den Verkehr umumgänglich und abgesehen von der geografischen Lage von Gebäuden, Straßen usw. ist auch das Verhalten der Menschen ein wichtiger Bestandteil.
Um diese Menschen effektiv simulieren zu können, bedarf es Spezifikation und Konfiguration geografischer Daten um ein verhalten von Menschen berechenbar zu machen.
Hier kommt die (NAME DER SOFTWARE) ins spiel.
Mit Hilfe von (NAME DER SOFTWARE) soll es  Verkehrsingenieuren und Verkehrswissenschafter möglich sein, genau solche Daten für ihre Berechnungen anfertigen zu können.
Hierbei werden Daten aus der OSM als Basis genutzt.
Die Nutzer müssen nun kein Vorwissen an Informatik mit sich bringen um die Software bedienen zu können.
Die Software simplifiziert die OSM Daten von Gebäude auf Punkte. Der Nutzer kann diese mit Tags versehen und Kategorisieren über selber angegebene Filter.
Die Berechnung für Attraktivität soll Konfigurierbar sein und kann von beliebigen Attributen des Nutzers festgelegt werden.
Neben der Attraktivität einzelner Elemente kann auch über mehrere Aggregiert werden, was ebenfalls Konfigurierbar sein soll.

\subsection{Musskriterien}
\textbf{author=Simon,reviewedby=Karl\\}
Musskriterien: unabdingbare Leistungen der Software.

\textbf{reviewDone = True, author=Karl\\}
\begin{itemize}
    \item Der Nutzer muss OSM-Datensätze einlesen können
    \item Der Nutzer muss Kartenausschnitte auswählen können
    \item Der Nutzer muss OSM-Elemente mithilfe von Black- und Whitelisten zu Kategorien zusammenfassen können
    \item Das Programm muss Standard-Kategorien anbieten
    \item Das Programm muss OSM-Elemente auf einen Punkt reduzieren können
    \item Das Programm muss den OSM-Elementen Attribute zuweisen können
    \item Das Programm muss die Zwischenschritte der Berechnung und Konfigurationen speichern und laden können
    \item Der Nutzer muss für jede Kategorie das Verfahren zur Berechnung der Attribute auswählen können
    \item Der Nutzer muss Standardwerte für Attribute festlegen können
    \item Der Nutzer muss Attraktivitätsattribute je Kategorie in Abhängigkeit zu Attributen definieren können
    \item Das Programm muss in jeder Verkehrszelle die Attraktivitätsattribute aggregieren können
    \item Es muss eine GUI geben
    \item Das Programm soll den Nutzer warnen und anzeigen können, welche Etappen neu berechnet werden müssen, bei Änderungen von Konfigurationen
\end{itemize}


\subsection{Wunschkriterien}
\textbf{author=Simon, reviewedBy=Karl, reviewDone = True\\}
\begin{itemize}
    \item Der Nutzer kann OSM-Daten per Link oder OSM-Repository einlesen
    \item Die Verkehrsdaten können mit einer Karte visualisiert werden
    \item Der Nutzer kann durch Klicken auf eine Karte Polygone definieren
    \item Es kann für das Filtern mit Kategorien beliebige logische Ausdrücke möglich sein
    \item Das Programm kann Standard-Attribute vorgeschlagen, für die Kategorie Erstellung
\end{itemize}

\subsection{Abgrenzungskriterien}
\begin{itemize}
    \item Es findet keine Analyse der generierten Daten statt
\end{itemize}
\newpage



\section{Produkteinsatz}

\subsection{Anwendungsbereiche}
\textbf{author="JP", reviewedby=Karl} 
Dieses Kapitel erläutert, in welchen Bereichen die Software eingesetzt werden soll. Als Open-Source-Veröffentlichung soll die Anwendung Verkehrsinstituten dem Berechnen von Daten dienen, welche anschließend als Eingabe zum Erstellen von Verkehrsnachfragemodellen verwendet werden. Diese sind später die Grundlage, um sowohl verkehrsplanerische, wie auch politische Entscheidungen bezüglich der Infrastruktur zu treffen.

\subsection{Zielgruppen}
\textbf{reviewedby = Felix, author=Karl\\}
Dieser Abschnitt definiert die Zielgruppe des Projekts. Prinzipiell kann jede Person die Anwendung nutzen. Nutzer müssen jedoch Grundkenntnisse im OSM-Dateiformat besitzen. Außerdem sind die Anwendungsbereiche, wie oben beschrieben, beschränkt. Daher bilden Verkehrsingenieure und Verkehrswissenschaftler die größte Zielgruppe des Produkts. Da die Zielgruppe insbesondere nicht aus Informatikern besteht, werden keine Kenntnisse in der Programmierung vorausgesetzt.

\subsection{Betriebsbedingungen}
\textbf{review=Simon, auhor=Pascal\\}
Dieses Unterkapitel behandelt die Bedürfnisse und Anforderungen der Software.
Zum Verwenden der Software werden Daten aus der OpenStreetMap benötigt, spezifischer sollen .osm.pbf / .osm.bz2 Daten von der Software einlesbar sein, ebenso wie Dateien, welche die Software selber wieder exportieren kann.

\newpage
Die Software arbeitet in Etappen, welche definiert sind durch
\begin{enumerate}
    \item Dateninput
    \item Geofilter
    \item Tag Filter
    \item Reduktion
    \item Attraktivität
    \item Aggregation
\end{enumerate}

(Für Spezifikation dieser Etappen siehe HIER KONTEXT ANGEBEN)
\newline
Diese Etappen werden nach einander ausgeführt und bedarf gegebenenfalls Konfiguration durch den Benutzer, bevor die Berechnungen ausgeführt werden kann.
Es sollen dabei Zwischenergebnisse der einzelnen Etappen gespeichert werden.
Die Software soll nun auch in der Lage sein, selber ihre Konfigurationen als Datei exportieren zu können (DATEITYP ANGEBEN NOCH). Dies muss allerdings keine vollständig berechnete Datei sein, sondern diese kann auch nur bis zu einer gewissen Etappe erfolgt sein.
Diese von der Software exportierten Dateien sollen ebenfalls wieder importierbar sein, um an diesen weiter arbeiten zu können, oder um mehreren Nutzern an verschiedenen Geräten die Selbe Datei zur Verfügung stellen zu können.
Innerhalb der Software soll es möglich sein auch ab einer gewissen Etappe neu Berechnungen mit anderen Konfigurationen anzustoßen, hierbei wird angezeigt welche Etappen alle betroffen sind und wo ebenfalls, gegebenenfalls Konfigurationen neu angegeben werden müssen.
Die Software soll auf dem Betriebssystem Windows 10 oder 11 laufen.
Dabei soll für kleinere Instanzen an Daten ein Laptop mit 8GB RAM, i5 4 Kerne ausreichen.
Eine kleine Instanz ist definiert als eine osm Datei, welche nicht mehr als 250mB an Speicher verbraucht.
Für größere Instanzen wird eine Workstation mit 125-250 GB RAM, 8-16 Kerne benötigt.
Eine Große Instanz ist definiert als alles, was keine kleine Instanz ist.
Für niederwertigere Hardware wird keine Funktion gewährleistet.
Die Software soll dabei von einer Person bedienbar sein, ohne jegliches Vorwissen an Informatik.

\newpage



\section{Produktübersicht}
\newpage



\section{Produktfunktionen}
\newpage



\section{Produktdaten}
\newpage



\section{Nichtfunktionale Anforderungen}

Dieses Kapitel definiert die nichtfunktionalen Anforderungen und Qualitätsmerkmale der Software.

Reminder: Punktreduktion soll erweiterbar sein



\subsection{Funktionalität, author=Pascal}

\begin{tabular}{|l| c| c| c| c|}
    \hline
        Produktqualität & sehr gut & gut & normal & nicht relevant \\
    \hline
        Angemessenheit & & x & &\\
    \hline
        Richtigkeit & x & & &\\
    \hline
        Ordnungsmäßgkeit & x & & &\\
    \hline
        Interoperabilität & & & & x\\
    \hline
        
    \end{tabular}

\textbf{Angemessenheit}
\newline
Die Angemessenheit ist als gut einzustufen, da die gesamte Software für ihre Funktion geeignet sein muss und eine schwere Handhabung der Software nur zu unnötigem Aufwand seitens des Nutzers führt.

\textbf{Richtigkeit und Ordnungsmäßigkeit}
\newline
Da die von der Software generierten Daten für weitere Verwendungen benutzt werden und man ein deterministisches verhalten möchte bei gleichen Konfigurationen, ist eine sehr gute Richtigkeit und Ordnungsmäßigkeit der Software unumgänglich. Denn auf fehlerhaften Daten und oder undeterministisch generierte Daten, zu Arbeiten führt nur zu Folgefehlern welche zum verfehlen des eigentlichen Ziels führen.

\textbf{Interoperabilität}
\newline
Da die Software nur auf Windows 10/11 funktionieren soll und keine Kommunikation zwischen Geräten vorsieht ist eine Interoperabilität irrelevant.

\subsection{Sicherheit, author=Simon}

\textbf{Zuverlässigkeit}
Durch definitition von Testfällen im Pflichtenheft und das spätere ausführen von diesen und weiteren währen der Implementierungsphase, können Fehler früh zeit gefunden und behoben werden.
Dadurch können wir eine äquadate Zuverlässigkeit gewährleisten.

\textbf{Fehlertoleranz}
Trotz aller Testfälle, kann es zu Fehlern kommen während des Program läuft oder mit dem Endergebniss, durch korrumpierte Daten oder durch fehlerhafte Eingabe bei der User Eingabe(MAYBE WEGLASSEN, VLT. GEWÄHRLEISTEN WIR ALLE USER INPUTS).
Die Fehlertoleranz des Programmes variiert von Fall zu fall.

\textbf{wiederherstellbarkeit}
Jeder Zwischenschnitt wird nach seiner berechnung lokal auf dem Endgerät gepseichert, dass das Programm ausführt, kommt es zum absturz während eines rechnenschrittes so kann die wiedeherstellung(VERWEIS MUSSKRITERIUM) des zu berechenden schrittes nicht gewährleistet werden, andernsfalle ist  eine widerherstellung aller Zwischenschritte gewährleistet.


\subsection{Benutzbarkeit, author = Felix}

    \begin{tabular}{|l| c| c| c| c|}
    \hline
        Produktqualität & sehr gut & gut & normal & nicht relevant \\
    \hline
        Verständlichkeit & x & & &\\
    \hline
        Erlernbarkeit & & x & &\\
    \hline
        Bedienbarkeit & x & & &\\
    \hline
        Effizienz & & & x &\\
    \hline
        Zeitverhalten & & & x &\\
    \hline
        Verbrauchsverhalten & & x & &\\
    \hline
    \end{tabular}

\textbf{Verständlichkeit, Erlernbarkeit und Bedienbarkeit}\\
Die Anwendung soll intuitiv zugänglich sein, um ein reibungsloses Einlesen der Daten zu ermöglichen. Da die Anwendung auch an Nicht-Informatiker gerichtet ist, muss die Bedienung leicht einsehbar und verständlich sein.

\textbf{Effizienz, Zeitverhalten und Verbrauchsverhalten}\\
Die Effizienz der Anwendung ist als normal eingestuft, da kein schnelles Reaktionsverhalten in der Berechnung gefordert ist, trotzdem aber eine gewisse Benutzerfreundlichkeit erhalten bleiben werden muss. Folglich soll die Anwendung ein gutes Zeit- und Verbrauchsverhalten haben.

\subsection{Änderbarkeit}
    \begin{tabular}{|l| c| c| c| c|}
    \hline
        Produktqualität & sehr gut & gut & normal & nicht relevant \\
    \hline
        Analysierbarkeit & x & & &\\
    \hline
        Modifizierbarkeit & & x & &\\
    \hline
        Stabilität & x & & &\\
    \hline
        Prüfbarkeit & & & x &\\
    \hline
        Übertragbarkeit & & & x &\\
    \hline
        Anpassbarkeit & & x & &\\
    \hline
        Installierbarkeit & & x & &\\
    \hline
        Austauschbarkeit & & x & &\\
    \hline
    \end{tabular}
    
\textbf{Analysierbarkeit, Modifizierbarkeit, Anpassbarkeit und Austauschbarkeit}\\
Da Erweiterungen ein häufig gewünschtes Kriterium darstellt, ist das Produkt gut Modifizierbar und der Programmcode ist gut analysierbar. Um Änderungen umzusetzen ist jedoch Code-Verständnis erforderlich (Informatiker gefordert).

\textbf{Stabilität}\\
Die Stabilität des Produktes muss gegeben sein.

\textbf{Installierbarkeit}\\
Das Produkt ist als Open-Source-Anwendung online herunterzuladen



\subsection{Qualitätsanforderungen}


\newpage



\section{Benutzeroberfläche/Schnittstellen}
\newpage



\section{Technische Produktumgebung}

\subsection{Software}
\subsection{Hardware}
\subsection{Produktschnittstellen}

\section{Glossar}
\textbf{OSM}\\
OpenStreetMap ist ein freies Projekt, das frei nutzbare Geodaten sammelt, strukturiert und für die Nutzung durch jedermann in einer Datenbank vorhält.

\textbf{Verkehrsnachfragemodell}\\
Ein Verkehrsnachfragemodell ist ein Modell, das alle relevanten Entscheidungsprozesse der Menschen nachbildet, die zu Ortsveränderungen führen. Im Personenverkehr umfassen diese Entscheidungen die Aktivitätenwahl, die Zielwahl, die Verkehrsmittelwahl, die Abfahrtszeitwahl und die Routenwahl.

\textbf{Geofilter}\\
Tool zur Einteilung von Räumen auf der Karte.

\textbf{Tag}\\
Eigenschaft einer Node in OSM.

\textbf{Tag Filter}\\
Filterung nach Tags.

\textbf{Reduktion}\\
Verfahren zur Abbildung von Gebäuden auf Punkte mit Attraktivitäten.

\textbf{Attraktivität}\\
Eigenschaft eines Gebäudes.

\textbf{Aggregation}\\
Verfahren zum Zusammenfassen mehrerer Attraktivitäten innerhalb einer Verkehrszelle.

\textbf{Node}\\
Objekt in OSM.

\textbf{Repository}\\
Zentrale Ablage in der digitale Daten, Dokumente, Objekte und Programme mit ihren Metadaten verwaltet werden.

\textbf{Verkehrszelle}\\
Bezeichnet einen räumlich abgegrenzten Teil eines Untersuchungsgebietes, der als Bezugseinheit bei der Verkehrsanalyse und Verkehrsprognose dient.

\newpage

\end{document}