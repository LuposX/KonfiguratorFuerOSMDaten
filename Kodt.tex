\documentclass[parskip=full]{scrartcl} % enables german formatting

\usepackage[utf8]{inputenc} % use utf8 file encoding for TeX sources
\usepackage[T1]{fontenc} % avoid garbled Unicode text in pdf
\usepackage[german]{babel} % german hyphenation, quotes, etc
\usepackage{hyperref} % detailed hyperlink/pdf configuration
\hypersetup{ % ‘texdoc hyperref‘ for options
pdftitle={PSE},
bookmarks=true,
}
\usepackage{csquotes} % provides \enquote{} macro for "quotes"

\title{Pflichtenheft}
\author{...}
\date{November 2022}

\begin{document}




\maketitle
\newpage

\tableofcontents
\newpage




\section{Zielbestimmung}

\textbf{author=Pascal,not reviewed yet\\}
Verkehrsmodelle sind allgegenwärtig, egal ob man nur ein Teilnehmer am Verkehr ist oder Entscheidungen treffen muss um diesen zu optimieren.
Für eine effektive Optimierung des Verkehrswesens sind Daten über den Verkehr umumgänglich und abgesehen von der geografischen Lage von Gebäuden, Straßen usw. ist auch das Verhalten der Menschen ein wichtiger Bestandteil.
Um diese Menschen effektiv simulieren zu können, bedarf es Spezifikation und Konfiguration geografischer Daten um ein verhalten von Menschen berechenbar zu machen.
Hier kommt die (NAME DER SOFTWARE) ins spiel.
Mit Hilfe von (NAME DER SOFTWARE) soll es  Verkehrsingenieuren und Verkehrswissenschafter möglich sein, genau solche Daten für ihre Berechnungen anfertigen zu können.
Hierbei werden Daten aus der OSM als Basis genutzt.
Die Nutzer müssen nun kein Vorwissen an Informatik mit sich bringen um die Software bedienen zu können.
Die Software simplifiziert die OSM Daten von Gebäude auf Punkte. Der Nutzer kann diese mit Tags versehen und Kategorisieren über selber angegebene Filter.
Die Berechnung für Attraktivität soll Konfigurierbar sein und kann von beliebigen Attributen des Nutzers festgelegt werden.
Neben der Attraktivität einzelner Elemente kann auch über mehrere Aggregiert werden, was ebenfalls Konfigurierbar sein soll.

\subsection{Musskriterien}
Musskriterien: unabdingbare Leistungen der Software.

\begin{itemize}
    \item Der Nutzer muss OSM-Datensätze einlesen können.
    \item Der Nutzer muss Kartenausschnitte auswählen können.
    \item Der Nutzer muss OSM-Elemente mithilfe von Black- und Whitelisten zu Kategorien zusammenfassen können.
    \item Das Programm muss Standard-Kategorien anbieten.
    \item Das Programm muss OSM-Elemente auf einen Punkt reduzieren können.
    \item Das Programm muss den OSM-Elementen Attribute zuweisen können.
    \item Der Nutzer muss für jede Kategorie das Verfahren zur Berechnung der Attribute auswählen können.
    \item Der Nutzer muss Standardwerte für Attribute festlegen können.
    \item Der Nutzer muss Attraktivitätsattribute je Kategorie in Abhängigkeit zu Attributen definieren können.
    \item Das Programm muss in jeder Verkehrszelle die Attraktivitätsattribute aggregieren können.
    \item Es muss eine grafische Oberfläche geben.
    \item Das Programm muss die Zwischenschritte der Berechnungen und Konfigurationen speichern und laden können.
    \item Der Nutzer muss die Berechnungen aus jeder Phase aus starten können.
    \item Der Nutzer muss Projekte erstellen können
\end{itemize}


\subsection{Wunschkriterien}
\textbf{author=Simon, reviewedBy=Karl, reviewDone = True\\}
\begin{itemize}
    \item Der Nutzer kann OSM-Daten per Link oder OSM-Repository einlesen.
    \item Die Verkehrsdaten können mit einer Karte visualisiert werden.
    \item Der Nutzer kann durch Klicken auf eine Karte Polygone definieren.
    \item Es kann für das Filtern mit Kategorien beliebige logische Ausdrücke möglich sein.
    \item Das Programm kann Attribute vorgeschlagen, für die Kategorie Erstellung.
    \item Das Programm kann die Ergebnisse der Aggregation grafisch darstellen
    \item Die Anwendung markiert Phasen, falls sie vom Nutzer abgeändert wurden
    
\end{itemize}

\subsection{Abgrenzungskriterien}
\begin{itemize}
    \item Es findet keine Analyse der generierten Daten statt.
\end{itemize}
\newpage

\section{Produkteinsatz}

\subsection{Anwendungsbereiche}
\textbf{author="JP", reviewedby=Karl} 
Dieses Kapitel erläutert, in welchen Bereichen das Produkt eingesetzt werden soll. Als Open-Source-Veröffentlichung soll die Anwendung Verkehrsinstituten dem Berechnen von Daten dienen, welche anschließend als Eingabe zum Erstellen von Verkehrsnachfragemodellen verwendet werden. Diese sind später die Grundlage, um sowohl verkehrsplanerische, wie auch politische Entscheidungen bezüglich der Infrastruktur zu treffen.


\subsection{Zielgruppen}
\textbf{reviewedby = Felix, JP author=Karl\\}
Dieser Abschnitt definiert die Zielgruppe des Projekts. Prinzipiell kann jede Person die Anwendung nutzen. Nutzer müssen jedoch Grundkenntnisse im OSM-Dateiformat besitzen. Die Anwendungsbereiche sind, wie oben beschrieben, beschränkt. Daher bilden Verkehrsingenieure und Verkehrswissenschaftler die größte Zielgruppe des Produkts. Da die Zielgruppe insbesondere nicht aus Informatikern besteht, werden keine Kenntnisse in der Programmierung vorausgesetzt.


\newpage







\section{Produktübersicht}
\newpage







\section{Produktfunktionen}
\newpage







\section{Produktdaten}
\newpage







\section{Nichtfunktionale Anforderungen}

Dieses Kapitel definiert die nichtfunktionalen Anforderungen und Qualitätsmerkmale der Software.

Reminder: Punktreduktion soll erweiterbar sein



\subsection{Funktionalität, author=Pascal, reviewedby=Karl}

\begin{tabular}{|l| c| c| c| c|}
    \hline
        Produktqualität & sehr gut & gut & normal & nicht relevant \\
    \hline
        Angemessenheit & & x & &\\
    \hline
        Richtigkeit & x & & &\\
    \hline
        Ordnungsmäßgkeit & x & & &\\
    \hline
        Interoperabilität & & & & x\\
    \hline
        
    \end{tabular}

\textbf{Angemessenheit}
\newline
Die Angemessenheit ist als gut einzustufen, da die gesamte Software für ihre Funktion geeignet sein muss und eine schwere Handhabung der Software nur zu unnötigem Aufwand seitens des Nutzers führt.



\textbf{Richtigkeit und Ordnungsmäßigkeit}
\newline
Die Aufgabe der Software ist über die Konfigurationen des Nutzers und den eingelesenen Dateien, neue Daten zu berechnen, welche weiter verwendet werden können. Hierbei ist es wichtig, dass diese Daten auch korrekt berechnet werden und ein deterministisches Verhalten gewährleistet ist.
Daher ist die Richtigkeit und Ordnungsmäßigkeit der Software als sehr gut einzustufen.

\textbf{Interoperabilität}
\newline
Es ist keine Kommunikation zwischen Geräten vorgesehen, daher ist die Interoperabilität als irrelevant einzustufen.


% remove this if we dont want to have a new page for each
% subsection
\newpage 

\subsection{Sicherheit, author=Simon}

    \begin{tabular}{|l| c| c| c| c|}
    \hline
        Produktqualität & sehr gut & gut & normal & nicht relevant \\
    \hline
        Zuverlässigkeit & x & & &\\
    \hline
        Fehlertoleranz & & x & &\\
    \hline
        Wiederherstellbarkeit & & x & &\\
    \hline
     \end{tabular}

\textbf{Zuverlässigkeit\\}
Die Zuverlässigkeit ist wichtig, da Benutzer erwarten, dass die Anwendung konsistente Ergebnisse liefert.
Es wird erwartet, dass das Programm die eingegebenen Daten basierend auf Benutzereingaben korrekt auswertet.
So muss das Programm immer das gleiche Endergebnis mit den gleichen Daten und Benutzereingaben zurückgeben.
Durch das Definieren von Testfällen in dem Pflichtenheft und das Ausführen dieser Testfälle während der Implementierungsphase können Fehler frühzeitig erkannt und behoben werden.
Dies gewährleistet eine sehr gute Zuverlässigkeit.

\textbf{Fehlertoleranz\\}
Trotz aller Testfälle, kann es zu Fehlern kommen während das Programm läuft. Diese Fehler können Entstehen durch korrumpierte Daten oder durch fehlerhafte Eingabe des Users.
Fehler die durch falsche Eingabe des Users Entstehen können abgefangen werden und der User benachrichtigt werden, somit kann dieser die fehlerhafte Eingabe korrigieren.
Fehler die ausgelöst werden durch korrumpierte Daten, können behoben werden indem der User benachrichtigt wird und dieser die korrumpierte Daten ersetzt.
Da diese Fehler nur enstehen können durch das falsche Verhalten des Users, kann 
die Fehlertoleranz als gut eingestuft werden.

\textbf{Wiederherstellbarkeit\\}
Beim Erstellen der Konfigurationen kann der Nutzer zu jedem Zeitpunkt die Konfigurationen lokal zwischenspeichern. Diese kann er zu einem späteren Zeitpunkt wieder einlesen. Dadurch wird gewährleistet, dass die Konfigurationsentscheidungen des Nutzers einfach wiederhergestellt werden können.
Auch alle Zwischenergebnisse, der vom Programm durchgeführten Berechnungen, werden direkt nach Fertigstellung auf dem Endgerät gespeichert. Während der einzelnen Berechnungen sind jedoch keine Zwischenspeicherungen vorgesehen.
Somit ist die die Wiederherstellbarkeit als gut einzustufen.


% remove this if we dont want to have a new page for each
% subsection
\newpage 


\subsection{Benutzbarkeit, author = Felix}

    \begin{tabular}{|l| c| c| c| c|}
    \hline
        Produktqualität & sehr gut & gut & normal & nicht relevant \\
    \hline
        Verständlichkeit & x & & &\\
    \hline
        Erlernbarkeit & & x & &\\
    \hline
        Bedienbarkeit & x & & &\\
    \hline
        Effizienz & & & x &\\
    \hline
        Zeitverhalten & & & x &\\
    \hline
        Verbrauchsverhalten & & x & &\\
    \hline
    \end{tabular}

\textbf{Verständlichkeit, Erlernbarkeit und Bedienbarkeit}\\
Die Anwendung soll intuitiv zugänglich sein, um ein reibungsloses Einlesen der Daten zu ermöglichen. Da die Anwendung auch an Nicht-Informatiker gerichtet ist, muss die Bedienung leicht einsehbar und verständlich sein. Dies erfolgt über eine moderne GUI.

\textbf{Effizienz, Zeitverhalten und Verbrauchsverhalten}\\
Die Effizienz der Anwendung ist als normal eingestuft, da kein schnelles Reaktionsverhalten in der Berechnung gefordert ist, trotzdem aber eine gewisse Benutzerfreundlichkeit erhalten bleiben werden muss. Folglich soll die Anwendung ein gutes Zeit- und Verbrauchsverhalten haben.
Eine kleine Instanz ist auf einem Laptop innerhalb von zwei Stunden berechenbar. Für größere Instanzen wird ein stärkerer Computer empfohlen.


% remove this if we dont want to have a new page for each
% subsection
\newpage 


\subsection{Änderbarkeit, author = JP, reviewedBy = Felix}
    \begin{tabular}{|l| c| c| c| c|}
    \hline
        Produktqualität & sehr gut & gut & normal & nicht relevant \\
    \hline
        Analysierbarkeit & x & & &\\
    \hline
        Modifizierbarkeit & & x & &\\
    \hline
        Stabilität & x & & &\\
    \hline
        Prüfbarkeit & & & x &\\
    \hline
        Übertragbarkeit & & & x &\\
    \hline
        Anpassbarkeit & & x & &\\
    \hline
        Installierbarkeit & & x & &\\
    \hline
        Austauschbarkeit & & x & &\\
    \hline
    \end{tabular}
    
\textbf{Analysierbarkeit, Modifizierbarkeit, Anpassbarkeit und Austauschbarkeit}\\
Erweiterungen stellen ein häufig gewünschtes Kriterium dar. Deshalb ist das Produkt gut modifizierbar. Der Programmcode ist gut analysierbar. Um Änderungen umzusetzen ist jedoch Code-Verständnis erforderlich (Informatiker gefordert).

\textbf{Stabilität}\\
Die Stabilität des Produktes muss gegeben sein.

\textbf{Installierbarkeit}\\
Das Produkt ist als Open-Source-Anwendung online herunterzuladen


% remove this if we dont want to have a new page for each
% subsection
\newpage 


\subsection{Qualitätsanforderungen, author = Karl}
In diesem Abschnitt werden die oben genannten Qualitätseigenschaften als konkrete Qualitätsanforderungen formuliert.

\begin{itemize}
    \item Die Anwendung soll in englischer Sprache verfügbar sein
    \item Die Anwendung soll intuitiv bedienbar sein
    \item Die Anwendung muss auf Windows 10 und 11 nutzbar sein
    \item Durch eine große Testabdeckung soll die Korrektheit des Produkts gewährleistet sein
    \item Das Produkt soll um neue Verfahren erweiterbar sein
    \item Das Produkt muss als Open-Source-Projekt weiterentwickelbar sein
    \item Der Quelltext soll lesbar und kommentiert sein
    \item Berechnungszeiten sollen minimal sein
    \item Die Programmiersprache muss Python sein.
\end{itemize}

\newpage



\section{Benutzeroberfläche/Schnittstellen}
\newpage



\section{Technische Produktumgebung, author = JP}
In diesem Kapitel wird die technische Umgebung des Produktes beschrieben.

\subsection{Software}
\begin{itemize}
    \item Implementierungssprache der Anwendung – Python
    \item Implementierungssprache der Benutzeroberfläche - Python (?)
\end{itemize}

\subsection{Hardware}
\begin{itemize}
    \item Windows 10/11 Laptop mit 8 GB RAM und i5 4 Kerne (8 log.) für kleine Instanzen
    \item Workstation mit 125 – 250 GB RAM und 8-16 Kerne für große Instanzen Effiziente
\end{itemize}

\subsection{Produktschnittstellen}
Die Benutzerschnittstelle wird über ein GUI zur Verfügung gestellt.

\newpage
\section{Glossar}
\textbf{OSM}\\
OpenStreetMap ist ein freies Projekt, das frei nutzbare Geodaten sammelt, strukturiert und für die Nutzung durch jedermann in einer Datenbank vorhält.

\textbf{Verkehrsnachfragemodell}\\
Ein Verkehrsnachfragemodell ist ein Modell, das alle relevanten Entscheidungsprozesse der Menschen nachbildet, die zu Ortsveränderungen führen. Im Personenverkehr umfassen diese Entscheidungen die Aktivitätenwahl, die Zielwahl, die Verkehrsmittelwahl, die Abfahrtszeitwahl und die Routenwahl.

\textbf{Geofilter}\\
Tool zur Einteilung von Räumen auf der Karte.

\textbf{Tag}\\
Eigenschaft einer Node in OSM.

\textbf{Tag Filter}\\
Filterung nach Tags.

\textbf{Reduktion}\\
Verfahren zur Abbildung von Gebäuden auf Punkte mit Attraktivitäten.

\textbf{Attraktivität}\\
Eigenschaft eines Gebäudes.

\textbf{Aggregation}\\
Verfahren zum Zusammenfassen mehrerer Attraktivitäten innerhalb einer Verkehrszelle.

\textbf{Node}\\
Objekt in OSM.

\textbf{Repository}\\
Zentrale Ablage in der digitale Daten, Dokumente, Objekte und Programme mit ihren Metadaten verwaltet werden.

\textbf{Verkehrszelle}\\
Bezeichnet einen räumlich abgegrenzten Teil eines Untersuchungsgebietes, der als Bezugseinheit bei der Verkehrsanalyse und Verkehrsprognose dient.

\newpage

\end{document}